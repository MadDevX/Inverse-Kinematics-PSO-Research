% !TeX spellcheck = en_US
\documentclass[]{report}
\usepackage{etoolbox}
\usepackage[utf8]{inputenc}
\usepackage{geometry}
\usepackage{color,soul}
\usepackage{sidecap}
\usepackage{blindtext}
\usepackage{amsmath}
\usepackage{wrapfig}
\usepackage{csvsimple}
\usepackage[shortlabels]{enumitem}
\usepackage{tabularx}
\usepackage{makecell}
\usepackage{graphicx}
\usepackage{booktabs}
\usepackage{hyperref}
\usepackage{svg}
\graphicspath{{pics/}{../pics/}}

\begin{document}
\newgeometry{tmargin=2cm, bmargin=2cm, lmargin=1.6cm, rmargin=1.6cm}
\title{Report from running Inverse Kinematics using PSO algorithm \\
\large Experiment 3 \\~\\~\\
\large Course: Research project - GPU algorithms \\
\large Coordinator: Krzysztof Kaczmarski }
\author{Eryk Dobrosz \\
	Marcin Gackowski}

{\let\newpage\relax\maketitle}

\chapter*{Description}
Project tackles inverse kinematics problem and solves it using Particle Swarm Optimization algorithm implemented with CUDA. Main goals of the project are:

\begin{itemize}
	\item real-time solving of inverse kinematics problem for arbitrarily large kinematic chains with many effectors
	\item solution choice that can be used for animation
\end{itemize}

\section*{Output Files}
\begin{itemize}
\item IK-diagnostics-positions.txt - contains positions of nodes each frame during test cases
\item IK-diagnostics-degrees.txt - contains rotations of nodes each frame during test cases
\item IK-diagnostics-distance.txt - contains aggregated distance of all effectors from their respective targets each frame during test cases
\item IK-diagnostics-frames.txt - contains number of frames required to reach acceptable solution for each test case
\end{itemize}

\chapter*{Report Goals}
\begin{itemize}
	\item Test implemented changes to fitness function.
	\item Detect whether system jittering still occurs and find which additional factors could be used by fitness function.
\end{itemize}


\chapter*{Computational Method}
\noindent Current iteration uses the same PSO algorithm as previous one with the exception of used fitness function and simulation space constraints. \\

\noindent As opposed to only taking into account distance from targets, currently used fitness function uses also latest kinematic chain configuration. For given coordinates in simulation space, function evaluates differences from latest configuration in each degree of freedom and adds their absolute value multiplied by arbitrary factor to the returned value. This approach ensures that local solutions will be chosen over more distant ones, even though choosing the later could lead to slight improvements in terms of distance to targets.\\

\noindent Another change concerns particle simulation which now uses angle constraints specified in given kinematic chain. Each angle constraint corresponds to either maximum or minimum value possible in given dimension. After each simulation step and before using fitness function, particle position is corrected to fit in the allowed space, thus discarding all solutions that did not met defined angle constraints.\\

\noindent Additionally, fitness function in current iteration also is responsible for collision detection. GJK algorithm is executed for each node-collider pair and if collision is detected examined solution is be discarded entirely and will not be considered in further calculations.

\chapter*{Description of the results}

\noindent Results were gathered through numerous executions of set test case. System configuration for testing matches test case used in previous iteration:

\begin{itemize}
	\item Kinematic chain with 21 degrees of freedom, that consists of:
	\begin{itemize}
		\item origin node placed in (0, 0, 0)
		\item 4 consecutive nodes creating a chain with all links of length equal to 1
		\item 3 effectors linked to the last node in the chain with links of length equal to 1
	\end{itemize}
	\item 16384 simulated particles
	\item 3 targets in set positions that can be reached by all effectors
	\item aggregated error threshold equal to 0.025 (sum of distances from effectors to their targets)
\end{itemize}

\noindent At the start of each test case system was reset to its default state, providing reproducible initial conditions for the algorithm.\\

\noindent Gathered results allow to verify used algorithm both in terms of performance and 
stability in approximation of optimal solution. Results for current iteration are as follows:

\begin{itemize}
	\item frames to reach satisfactory solution: 
	\begin{itemize}
		\item average: $33.1$ 
		\item min: $11$
		\item max: $171$
	\end{itemize}
	\item frame-difference for each degree of freedom (angle in radians): 
	\begin{itemize}
		\item average: $0.024$
		\item min: $0.00$
		\item max: $0.15$
	\end{itemize}
	\item frame-difference for each node position (distance in OpenGL units):
	\begin{itemize}
		\item  average: $0.022$ 
		\item min: $0.00$ 
		\item max: $0.12$
	\end{itemize}
\end{itemize}


\chapter*{Remarks}
\noindent Results from test case in this iteration show significant improvements in average rotation and position difference that present as follows when compared to previous iteration:
\begin{itemize}
	\item 8 times smaller rotation difference on average
	\item 5 times smaller position difference on average
	\item 8 times more frames needed to reach satisfactory solution on average
\end{itemize}

\noindent Significant increase in average frame count to reach solution is the result of using rotation differences to influence returned fitness function value, however current process of reaching desired solution much more resembles animation with almost no noticeable system jitters.\\

\noindent In this iteration even maximum rotation and position differences are comparable to average differences gathered for previous iteration of the project. This means that algorithm does not suffer from random jumps to entirely separate solution areas and is significantly better suited for use in animations.

\chapter*{Future works}
\begin{itemize}
\item Use position differences as an additional factor in choosing solutions.
\item Optimize fitness function to take advantage of GPU parallelism.
\end{itemize}

\end{document}